\chapter{Conclusion / Discussion}
\label{ch:Conclusion}

This reported aimed at bringing together two democratizing and revolutionizing concepts for a more active participation of members in the society within institutions of the society. For the institution of political process, this was addressed by liquid democracy, and for the institution of science, the concept of citizen science was used. These concepts exhibited significant synergies, which were addressed in the design for a \tracknshrink{CS}-enabled \tracknshrink{LD} platform. 

Based on theoretical analysis of the concepts both the conceptual design, as well as its engineering process was addressed. While an implementation of the platform fell outside the scope of this report, the specification of the platform was done to a degree that future work can seamlessly take up on the concepts developed here. 

\todo{Address core insights derived in theory}

\todo{Address core insights derived in concept design}

\todo{Address core insights derived in engineering part}

The following will further discuss the \nameref{sec:DiscussionImplementation}, and the state of the software, as well as putting the project into the scientific context (see \nameref{sec:DiscussionScientificContext}).

We conclude with a discussion on \nameref{sec:FutureWork}, or processes that any true project aiming for bringing together Citizen Science and Liquid Democracy in an empowering platform need to be able to implement.

\section{Discussion of Prototype Implementation}
\label{sec:DiscussionImplementation}

\section{Verortung im wissenschaftlichen Kontext (CS)}
\label{sec:DiscussionScientificContext}

Verweis auf 2.3

\section{Future work}
\label{sec:FutureWork}

The scope of this report was the specification and design of a \tracknshrink{CS}-enabled \tracknshrink{LD} platform. It thus cuts short of the implementation, and future work will be focused on establishing said platform. 

Implementation of the platform should involve a diverse range of actors, and should be oriented at their needs, preferences and practices. The model of political processes within the platform needs to respect how actors do and want to interact with one another and societal institutions, as well as paying attention to scientific and political processes. In-depth research and actor involvement should be addressed before and during development, and an agile approach should be employed for this. 

Further research should also be done into voting systems and their implementation into the platform, as well as into modern cryptological practices for ensuring the desired properties of the modeled processes.

Due to the diversity of the necessary processes (in particular actor involvement) and concepts, future work needs a capable and interdisciplinary team with strong technical and social skills, in order to bring into practice the ideas sketched in this report. Due to the importance of the questions addressed and its innovativeness we strongly believe there is a large potential for such a project, and believe that it can have a transformative influence on how science and  politics, in particular through the involvement of the actors a democratic system is based upon, are viewed and lived in the public discourse.