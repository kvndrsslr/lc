\chapter{Introduction}
\label{ch:Introduction}

\todo[inline]{FINAL TODOs:\\
- Search and Replace for CS and LD. Only the very first occurrence can be italics and capitalized, all other instances should be normal text\\
-- Citizen Science -> citizens science (no italics, just normal text)\\
-- Liquid Democracy -> liquid democracy \\
-- When using CS or LD make sure to wrap it in a \textbackslash{}tracknshrink\{CS/LD\}
}

While empowerment of the masses is central to the project of democracy and thus not a fundamentally new concept, the advent of digitalization brought with it tools for a new quality of emancipation. This is particularly true for political and scientific processes. Excellent examples for this are the concept of \textit{Liquid Democracy} (\tracknshrink{LD}) for the political sphere and the concept of \textit{Citizen Science} (\tracknshrink{CS}) for the scientific sphere. Due to similarities and synergies between these two concepts we are linking \tracknshrink{CS} and \tracknshrink{LD} in this project report. Both concepts share the same people-driven foundation, as both involve the participation of interested citizens and users likewise. Citizen science describes scientific research conducted by lay volunteers. In other words \textit{“Amateur Experts”}  \parencite{Gura2013}. The number of \tracknshrink{CS} projects and their uses in primary research have grown rapidly over the last years \parencite{Kosmala2016} and an increasing amount of today’s professional research relies on citizen scientists. Liquid democracy describes a compound of two fundamental types of democracy, namely (1) direct democracy and (2) representative or indirect democracy. That means, liquid democracy empowers citizens to either vote directly for a certain state of affairs or to  delegate their vote to another citizen. Additionally, both concepts share the same participatory nature, and despite being old ideas both of them eminently shine in the digital age. Interlinking these two conceptions is, however, a novel approach, and the primary objective of this project.

% This document is the project report of the  project group, a project in the \href{http://www.dh.uni-leipzig.de/wo/courses/summer-semester-20142015/citizen-science/}{\textit{Citizen Science}} course of the \href{http://www.dh.uni-leipzig.de/wo/}{department of Digital Humanities} at the University Leipzig in the summer semester of 2017.

\section{Motivation}
\label{sec:Motivation}

Participants of citizen science programs take an active role in a scientific research process. They usually work with other fellow citizen scientists towards the same goal and are often supervised---to some degree---by scientists, labs or institutions who initially started the study. Ultimately, citizen science projects entail multiple advantages for all stakeholders. For one, some research questions are simply impossible to answer without having an active mass helping to conduct, measure and interpret the data. Besides, this form of ‘distributed intelligence’ is a salient feature for \tracknshrink{CS}. On the other hand, citizen scientist gain additional knowledge in their field from other fellow peers. They strengthen their skills in communication, reporting, and scientific practices. There is also a motivational difference between a citizen scientist and a professional one. Since citizen scientists work voluntary in their field of interest, they don’t need be convinced, as they already have an affection for it. This may also yield in a different attitude towards their work and duties, and one could argue this ultimately leads to higher willingness to work.

\todo{den Gebruach von 'also' verstehe ich hier nicht; worauf referenziert das?}Liquid democracy may also offer many opportunities for societies, whether for smaller communities, countries or for unions of states (e.g., the \tracknshrink{EU}). As---simply put---liquid democracy combines direct and representative democracy, it offers whole new ways for the democratic participation process. This drastic shift would grant every citizens a natural right be heard, as everyone could use their voice for a certain state of affairs.

Since both concepts fully leverage their strengths in a digital environment, combining these concepts would further allow a liquid democracy system that is open for a wide field of research. Possible research fields would contain: psychology, sociology, demographics, and, apparently, political science, especially psephology (such as voter transition analysis). \todo{Was ist psephology? Mir nicht geläufig, dem Leser vermutlich auch nicht, muss erläutert werden} Not only researchers would benefit from such a system but society as well, since it would change the entire process of decision-making in politics and the way we track political and historical events. Additionally, the need for participation in societal processes becomes ever more apparent in the light of the current developments of our political system. An increasing number of people feel disenfranchised, with little opportunity to make their voice heard on a political stage.


\section{Objective}
\label{sec:Objective}

When reviewing different platforms we focused on free and open-source software (\tracknshrink{FOSS}) solutions, since this would allow to modify and extend functionality. Unfortunately, as of early 2019, there is no platform that is \tracknshrink{FOSS} and that also provides a substantial foundation for a \tracknshrink{LD} system where we could build upon. Most of the tools we inspected were sheer voting tools that have, obviously, no support for vote delegation, nor any option to access the system’s data. Since building an entire citizen science-enabled liquid democracy platform from scratch is beyond this project’s scope we will provide a conceptual framework instead. Thus, this work is structured as following manner.  

% These technological components need to enable the citizen to both participate in democratic processes in a way that respects its nature as mature political subject, as well as in enabling them to approach it as an object of research from a scientific perspective, even without being an expert in either the subject matter or research methods. 

% It thus needs to provide the functionalities to exhibit the processes of a fully-functional, comprehensive liquid democracy process, as well as the ease of access and learning resources, as well as research infrastructure to perform Citizen Science that deserves its name.




\section{Structure}
\label{sec:structure}

To get a good understanding of both underlying concepts we will introduce the the concepts of citizen science and liquid democracy in the following section. Thereafter, we formulate criteria for \tracknshrink{CS}-enabled \tracknshrink{LD} platform. We inspect related work and lay out a conceptual design. In section four we conceptualize our \textit{Liquid Citizen} platform, including software requirements, architectural design, etc. In the last section we will discuss our work and provide an outlook for future work that can be done. 
